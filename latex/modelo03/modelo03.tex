\documentclass{beamer}
\usepackage[portuguese]{babel}
\usepackage[utf8]{inputenc}
\usetheme{Madrid}


\title{Programação Orientada a Aspectos}

\subtitle{}

\author{Alexandre Castro\inst{1} \and Júlia Ribeiro\inst{2} \and Raphael\inst{3} \and Rallyson{4} \and Samuel\inst{5}\and Antônio\inst{6}}


\institute[Instituto Federal de Alagoas] 
{
  \inst{1}%
  Sistemas de informação\\
  Instituto Federal de Alagoas
 }


\date{30 de Agosto de 2018}


\subject{Theoretical Computer Science}

\AtBeginSubsection[]
{
  \begin{frame}<beamer>{Outline}
    \tableofcontents[currentsection,currentsubsection]
  \end{frame}
}


\begin{document}

\begin{frame}
  \titlepage
\end{frame}

\begin{frame}{Introdução}
  \tableofcontents
 
\end{frame}


\section{O que significa programação  orientada a aspectos}

\subsection{Conceitos Básicos}

\begin{frame}{O que é POA}{}
  \begin{itemize}
  
  
  \item {
   É um paradigma de programação de computadores que permite aos desenvolvedores de software separar e organizar o código de acordo com a sua importância para a aplicação (separation of concerns)
  }
  
  
  \item{A linguagem orientada a aspectos separa os interesses transversais(Crosscutting concerns) em módulos. Tais módulos são chamados de aspectos.
  }
  
  
  \end{itemize}
\end{frame}

\subsection{POO ou POA?}


\begin{frame}{POO ou POA?}
  \begin{itemize}
  
  
  \item {
     A POA não surgiu para substituir a POO, mas sim para trabalhar em conjunto.
    \pause % The slide will pause after showing the first item
  }
  \item {   
    Todo o programa escrito no paradigma orientado a objetos possui código que é alheio a implementação do comportamento do objeto. Este código é todo aquele utilizado para implementar funcionalidades secundárias e que se encontra espalhado por toda a aplicação (crosscutting concern). A POA permite que esse código seja encapsulado e modularizado.
  }
  
  \end{itemize}
\end{frame}

\section{Utilizando a POA}
\subsection{Joinpoint e Cutpoint}
\begin{frame}{Joinpoint e Cutpoint}
\begin{itemize}


    \item {Jointpoint são pontos de junção do programa que executam o programa.
    }
    \item { Pointcuts é um conjunto de pontos de junção e permite dizer onde exatamente aplicar o Advice.
    }
    \item {Advice é o código que deve ser aplicado em um determinado ponto do joinpoint de um programa.
    }
    
    
\end{itemize}

\end{frame}

\subsection{Interseção}
\begin{frame}{Interseção}

\begin{itemize}


    \item {O processo de interseção é conhecido como weaving.
                 Ele é responsável por combinar o código OO com o OA para a geração do sistema final.
          }
    
          
\end{itemize}

\end{frame}


\subsection{Organizando em POA}
\begin{frame}{Organizando em POA}

\begin{itemize}
    \item {Identificar e caracterizar requisitos \pause
    
    }
    \item {Implementar requisitos não-transversais utilizando CLASSES
    
    }
    \item { Implementar requisitos transversais utilizando ASPECTOS
    
    }
\end{itemize}
    
\end{frame}


% All of the following is optional and typically not needed. 
\appendix
\section<presentation>*{\appendixname}
\subsection<presentation>*{Para ler mais sobre}

\begin{frame}[allowframebreaks]
  \frametitle<presentation>{Para ler mais sobre}
    
  \begin{thebibliography}{10}
    
  \beamertemplatebookbibitems
  % Start with overview books.

  \bibitem{Author1990}
   Ricardo Terra (rterrabh [at] gmail.com) Outubro, 2013 Programação Orientada a Aspectos Ricardo Terra rterrabh [at] gmail.com Programação Orientada a Aspectos 1
  
  \beamertemplatearticlebibitems
  % Followed by interesting articles. Keep the list short. 


   
 
   
  \end{thebibliography}
\end{frame}

\end{document}


